\usepackage{cite}

Haskell - My Favorite Blub(1)
  * http://en.wikipedia.org/wiki/Haskell_(programming_language)
    * lazy
    * pure
    * functional
    * quality implementation(s)
      GHC is the standard implement. Others exist but do not offer the same feature set of GHC.
      * http://doi.acm.org/10.1145/2088456.1863528 "A systematic derivation of the STG machine
      verified in Coq"
  * An open standard for a functional language.
    * haskell 2010
    * haskell 98
  * Used as a base in a lot of functional language research.
    * type systems
    * evaluation systems
    * implementation performance
  * According to wikipedia: influenced 19+ languages
    * Scala
    * Perl 6
    * Visual Basic 9 (?!?)
    * CoffeeScript
    * plus a bunch of features in other languages.
  * GHC
    * the primary haskell system
  * install process
    * link to vagrant box
    * which I send out earlier
  * ghci
    * interactive prompt.
  * TOTALLY PRACTICAL STUFF
  * ghci commands - different from haskell code!
    * :help
  * :type ghci command
let x = 1
:type x
x :: Integer

  * Haskell syntax!
  * value language expression
    * "let" introduces a binding
    * "x" a binding identifier
    * "=" defines value the identifier is equal to
    * "1" the numeric value 1.
  * ":type x" <- not haskell syntax!
  * type language expression
    * "::" binary "is of type" operator
    * the lhs are identifiers
    * the rhs specifies the type.
      * the type language is not the value language
    * "Integer" the type identifier for arbitrary precision integer values.
  * We did not pair the definition with a type declaration. Haskell then inferred a most general
  (monomorphic) type for us.
  * If the most general type is not the type we want then the definition can be paired with a type declaration.
let x :: Int = 1
  * This is one method of pairing a definition with a declaration.
  * The type identifier "Int" is a fixed with integer type with good performance characteristics.
  * There are others...
    * Float
    * Double
    * Rational
    * 
  * wait... rationals?
    * yep!
    * works like you'd expect:
λ: x * 2 / 5 + 100
502 % 5
λ:
* polymorphism!
* :info
λ: :info Int
data Int = GHC.Types.I# GHC.Prim.Int#   -- Defined in `GHC.Types'
instance Bounded Int -- Defined in `GHC.Enum'
instance Enum Int -- Defined in `GHC.Enum'
instance Eq Int -- Defined in `GHC.Classes'
instance Integral Int -- Defined in `GHC.Real'
instance Num Int -- Defined in `GHC.Num'
instance Ord Int -- Defined in `GHC.Classes'
instance Read Int -- Defined in `GHC.Read'
instance Real Int -- Defined in `GHC.Real'
instance Show Int -- Defined in `GHC.Show'
λ:
  *
    * The type expression after the "=>" 
    * The type expression before the "=>" is the context.
    * In this case the context is a single constraint:
      "The type identified 
* :browse

1) http://en.wikipedia.org/wiki/Paul_Graham_(computer_programmer)#Blub

